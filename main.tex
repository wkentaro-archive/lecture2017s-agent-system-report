\documentclass[letterpaper, 11 pt, conference, onecolumn]{ieeeconf}  % Comment this line out if you need a4paper
%\documentclass[a4paper, 10pt, conference]{ieeeconf}      % Use this line for a4 paper

\IEEEoverridecommandlockouts                              % This command is only needed if 
                                                          % you want to use the \thanks command

\overrideIEEEmargins                                      % Needed to meet printer requirements.

% See the \addtolength command later in the file to balance the column lengths
% on the last page of the document

\usepackage[dvipdfmx]{graphicx}
\graphicspath{{figs/}}
\usepackage{amsmath}
\usepackage{amssymb}
\usepackage{algorithm}
\usepackage{algpseudocode}
\usepackage{times}
%
\newcommand{\figref}[1]{Fig.\ref{figure:#1}}
\newcommand{\tabref}[1]{Table \ref{table:#1}}
\newcommand{\argmax}{\operatornamewithlimits{argmax}}
\newcommand{\argmin}{\operatornamewithlimits{argmin}}

\title{\LARGE \bf
ヒューマンインターフェース講義レポート
}


\author{和田健太郎, 48-166636}
% \author{Albert Author$^{1}$ and Bernard D. Researcher$^{2}$% <-this % stops a space
% \thanks{*This work was not supported by any organization}% <-this % stops a space
% \thanks{$^{1}$Albert Author is with Faculty of Electrical Engineering, Mathematics and Computer Science,
%         University of Twente, 7500 AE Enschede, The Netherlands
%         {\tt\small albert.author@papercept.net}}%
% \thanks{$^{2}$Bernard D. Researcheris with the Department of Electrical Engineering, Wright State University,
%         Dayton, OH 45435, USA
%         {\tt\small b.d.researcher@ieee.org}}%
% }


\begin{document}

\maketitle
\thispagestyle{empty}
\pagestyle{empty}


% AR/VRなどのメディア系ICT研究のうち、新規性のある出口型研究をデザインせよ

近年,e-learningが広く受け入れられており,
塾や予備校だけでなく公的教育機関においても活用されている.
% そこで本レポートではメディア系ICT研究として,e-learningによる講義における
% VR技術の応用を提案する.
e-learningによる講義は講師が黒板やノート等に板書を書き込む様子と
講師の発話を記録することによって行われ,
通常の講義と比べ受講者が自由な時間に受講できることや
受講者にとって適した回数や速度で受講できることが利点として挙げられる.
一方で実際に教室へ赴いて講義を受ける場合と比べた欠点としては,
実際に講義を受けているという現実感の欠如が考えられ,
それによる学習効率の低下が懸念される.
本レポートではこれらの問題を解決することを目的としたメディア系ICT研究として,
e-learningによる講義におけるVR技術の応用を提案する.


% -------------------------------------------------------------------------------------------------
% \begin{abstract}
% \input src/abst.tex
% \end{abstract}

% \input src/intro.tex
% \input src/procedure.tex
% \input src/math.tex
% \input src/howto.tex
% \input src/conclusion.tex

% \addtolength{\textheight}{-12cm}   % This command serves to balance the column lengths
%                                   % on the last page of the document manually. It shortens
%                                   % the textheight of the last page by a suitable amount.
%                                   % This command does not take effect until the next page
%                                   % so it should come on the page before the last. Make
%                                   % sure that you do not shorten the textheight too much.

% \input src/appendix.tex
% \input src/acknowledge.tex

% \bibliographystyle{junsrt}
% \bibliography{main}
% -------------------------------------------------------------------------------------------------

\end{document}
