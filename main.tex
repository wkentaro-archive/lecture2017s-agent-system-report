\documentclass[letterpaper, 12 pt, conference, onecolumn]{ieeeconf}  % Comment this line out if you need a4paper
%\documentclass[a4paper, 10pt, conference]{ieeeconf}      % Use this line for a4 paper

\IEEEoverridecommandlockouts                              % This command is only needed if 
                                                          % you want to use the \thanks command

\overrideIEEEmargins                                      % Needed to meet printer requirements.

% See the \addtolength command later in the file to balance the column lengths
% on the last page of the document

\usepackage[dvipdfmx]{graphicx}
\graphicspath{{figs/}}
\usepackage{amsmath}
\usepackage{amssymb}
\usepackage{algorithm}
\usepackage{algpseudocode}
\usepackage{times}
\usepackage[margin=1.1in]{geometry}
%
\newcommand{\figref}[1]{Fig.\ref{figure:#1}}
\newcommand{\tabref}[1]{Table \ref{table:#1}}
\newcommand{\argmax}{\operatornamewithlimits{argmax}}
\newcommand{\argmin}{\operatornamewithlimits{argmin}}

\title{\LARGE \bf
ヒューマンインターフェース レポート
}


\author{和田健太郎, 48-166636}
% \author{Albert Author$^{1}$ and Bernard D. Researcher$^{2}$% <-this % stops a space
% \thanks{*This work was not supported by any organization}% <-this % stops a space
% \thanks{$^{1}$Albert Author is with Faculty of Electrical Engineering, Mathematics and Computer Science,
%         University of Twente, 7500 AE Enschede, The Netherlands
%         {\tt\small albert.author@papercept.net}}%
% \thanks{$^{2}$Bernard D. Researcheris with the Department of Electrical Engineering, Wright State University,
%         Dayton, OH 45435, USA
%         {\tt\small b.d.researcher@ieee.org}}%
% }


\begin{document}

\maketitle
\thispagestyle{empty}
\pagestyle{empty}


% AR/VRなどのメディア系ICT研究のうち、新規性のある出口型研究をデザインせよ

近年,e-learningが広く受け入れられており,
塾や予備校だけでなく公的教育機関においても活用されている.
% そこで本レポートではメディア系ICT研究として,e-learningによる講義における
% VR技術の応用を提案する.
e-learningによる講義は講師が黒板やノート等に板書を書き込む様子と
講師の発話を記録することによって行われ,
通常の講義と比べ受講者が自由な時間に受講できることや
受講者にとって適した回数や速度で受講できることが利点として挙げられる.
一方で実際に教室へ赴いて講義を受ける場合と比べた欠点としては,
実際に講義を受けているという現実感と講義の相互作用性の欠如
による学習効率の低下が考えられる.
本レポートではこれらの問題を解決することを目的としたメディア系ICT研究として,
e-learningによる講義におけるVR技術の応用を提案する.

まず一つ目の問題である現実感の欠如に関しては,
3Dカメラによる映像の記録によって映像として撮影し,
ヘッドマウントディスプレイや3Dディスプレイなどで
再生することによって解決する.
しかし,通常のRGB-Dカメラから得られる映像は平面的な三次元情報(2.5D)であり,
現実世界のように見回すことの可能なモデル(3D)を生成することが必要である.
そこでこの部分における目標は,
``環境と物体の動きを計測し,動的な三次元モデルを生成する''
というものであり,センサによって得られる色や深度にはノイズを考慮して
如何に現実に近づけるかということが研究課題となる.
現状の技術として,センサの比較的近傍で単一物体の動的三次元モデルを作ることは
実現可能なものとなってきている.
これは,デプスのノイズを複数視点で計測することによって軽減するということと,
動的な物体に対してカメラと物体の移動量を推定することによって実現されている.
しかし物体のテクスチャも含めた三次元モデル生成には,
より高精度な移動量の推定が必要となり,
これは現状の技術だけでは実現できず,新たな技術開発が必要となる.

次に二つ目の問題である講義の相互作用性の欠如に関しては,
生徒の質問に対して回答可能な音声システムの構築によって解決することを目標とする.
授業における質問にはパターンが存在すると考えられるため,
同じ授業を多数の学生が受講する中でQ\&Aのデータベースを構築し,
そのデータを基に機械にパターンを学習させることで
同様の質問に対して自然な形で回答を返すことのできるシステムを構築する.
この部分における技術的な課題としては,特に生徒の質問の文字列の意味理解を如何にして
機械にさせるかということだと考える.
質問の意味を理解し,同様の意味の質問(Q)をデータベースから作成し,
対応する回答(A)を自然な文に変えて返すということが,
少量のQ\&Aペアから上のようなモデルを構築するためには必要である.
現状の技術では,質問を入力として回答を返すという学習モデルにおいて
質問そのものの意味を理解するというよりは,
質問と回答の大量なペアによって質問から回答へのマッピングを学習するということが
主目的となっており,同様のアプローチが生徒固有の質問に対して固有の回答を返すという
問題において適用可能であるかということは疑問が残る.
質問の意味理解と,回答の生徒への固有化が今回の研究課題となる.

以上のe-learningにおける二つの問題に対して研究課題を設定し,
それらを解決することによって受講生の学習効率を向上させることが本プロジェクトの目的である.
e-learningは今後ますます需要が高まると考えられ,その学習効率向上は必須であり,
動的な三次元映像技術と文の意味理解と回答の固有化も画像認識,機械学習分野における
重要な研究課題であるため,本プロジェクトは社会的,学術的に価値のあるものだと考える.

% -------------------------------------------------------------------------------------------------
% \begin{abstract}
% \input src/abst.tex
% \end{abstract}

% \input src/intro.tex
% \input src/procedure.tex
% \input src/math.tex
% \input src/howto.tex
% \input src/conclusion.tex

% \addtolength{\textheight}{-12cm}   % This command serves to balance the column lengths
%                                   % on the last page of the document manually. It shortens
%                                   % the textheight of the last page by a suitable amount.
%                                   % This command does not take effect until the next page
%                                   % so it should come on the page before the last. Make
%                                   % sure that you do not shorten the textheight too much.

% \input src/appendix.tex
% \input src/acknowledge.tex

% \bibliographystyle{junsrt}
% \bibliography{main}
% -------------------------------------------------------------------------------------------------

\end{document}
